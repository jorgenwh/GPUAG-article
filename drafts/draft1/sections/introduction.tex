\section*{Introduction}

Genotyping is a prominent application of the massive amounts of data provided by high-throughput sequencing. 
Recent developments in graph and alignment-free methods of genotyping have resulted in genotyping methods yielding competitive accuracies with significantly reduced time costs compared with their alignment-based counterparts.
Despite these strides, the cost of genotyping in terms of computing power is still high.
As a result, many genotyping tools are implemented using verbose, low-level programming languages such as C and C++ where the programmers need to have in-depth understanding of both hardware and low-level programming tools to create competitive solutions.
However, some recent graph and alignment-free genotypers perform most of their computation in the form of array-operations. 
This has led to a more desireable approach, namely using array-libraries such as NumPy \cite{numpy} to create performant solutions in Python where the programmer can enjoy the programmatic simplicity of Python as well as the speed of carefully optimized C code.
One such genotyping tool is the current state-of-the-art graph-based and alignment-free genotyper KAGE \cite{kage}, which is written in Python and heavily leverages NumPy.
We have taken KAGE as a basis and developed a GPU supported version, GKAGE, showing that genotyping tools performing most of their compute as array-operations can benefit greatly from GPU support.

