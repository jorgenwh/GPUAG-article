\section*{Implementation details}
Both CuCounter and CuStats are designed to be compatible with both NumPy and CuPy.
Both NumPy and CuPy are in fact prerequisites for CuCounter and CuStats, as the supported features in both packages currently only interact with NumPy and CuPy arrays.
All current features in both packages will accept both NumPy and CuPy arrays and will always return arrays of the same type as the input.
When NumPy arrays are used, both packages will handle copying of the underlying array data to and from the GPU.
Naturally, using CuPy arrays will result in significantly better performance, as their underlying data will already reside in GPU memory.

\subsection*{CuCounter's underlaying hashtable}
The underlaying hashtable used in the CuCounter's counter object follows a simple open addressing with linear probing scheme. 
When an array of 64-bit encoded kmers are provided, each kmer is inserted by a single block thread in a kernel call. 
The thread will deploy a simple linear probing scheme and search the hashtable for either an empty position, or a position occupied by the provided kmer.
In this search, the initial position $p_i$ for kmer $k$ is found by computing
\begin{equation}
  p_i=hash(k) \bmod c
\end{equation}
and every consecutive position $p_{c+1}$ following position $p_c$ is determined by computing
\begin{equation}
  p_{c+1}=(p_c+1) \bmod c
\end{equation}
where $hash$ is a murmur hash function and $c$ is the hashtable's capacity.
The search terminates when either $k$ or an empty slot is observed in the hashtable.
For both insertion (during initialization) and counting, CUDA atomic operators are used to ensure validity.

\subsection*{CuStats functionality}
The CuStats package is meant to be a collection of useful statistical functions used in (graph and alignment-free based) genotyping.
While these functions can be effortlessly implemented using NumPy, and thereby CuPy to leverage the GPU, their performance suffer because of the general-purpose nature of NumPy and CuPy.
For instance, when creating nested array-expressions, CuPy will often perform many temporary array allocations, and needlessly move data back and fourth between registers and global memory.
Both of these problems are circumvented in CuStats by implementing targeted kernels where the entirity of the expressions can be efficiently computed.
